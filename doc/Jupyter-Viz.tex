% generated by GAPDoc2LaTeX from XML source (Frank Luebeck)
\documentclass[a4paper,11pt]{report}

\usepackage{a4wide}
\sloppy
\pagestyle{myheadings}
\usepackage{amssymb}
\usepackage[utf8]{inputenc}
\usepackage{makeidx}
\makeindex
\usepackage{color}
\definecolor{FireBrick}{rgb}{0.5812,0.0074,0.0083}
\definecolor{RoyalBlue}{rgb}{0.0236,0.0894,0.6179}
\definecolor{RoyalGreen}{rgb}{0.0236,0.6179,0.0894}
\definecolor{RoyalRed}{rgb}{0.6179,0.0236,0.0894}
\definecolor{LightBlue}{rgb}{0.8544,0.9511,1.0000}
\definecolor{Black}{rgb}{0.0,0.0,0.0}

\definecolor{linkColor}{rgb}{0.0,0.0,0.554}
\definecolor{citeColor}{rgb}{0.0,0.0,0.554}
\definecolor{fileColor}{rgb}{0.0,0.0,0.554}
\definecolor{urlColor}{rgb}{0.0,0.0,0.554}
\definecolor{promptColor}{rgb}{0.0,0.0,0.589}
\definecolor{brkpromptColor}{rgb}{0.589,0.0,0.0}
\definecolor{gapinputColor}{rgb}{0.589,0.0,0.0}
\definecolor{gapoutputColor}{rgb}{0.0,0.0,0.0}

%%  for a long time these were red and blue by default,
%%  now black, but keep variables to overwrite
\definecolor{FuncColor}{rgb}{0.0,0.0,0.0}
%% strange name because of pdflatex bug:
\definecolor{Chapter }{rgb}{0.0,0.0,0.0}
\definecolor{DarkOlive}{rgb}{0.1047,0.2412,0.0064}


\usepackage{fancyvrb}

\usepackage{mathptmx,helvet}
\usepackage[T1]{fontenc}
\usepackage{textcomp}


\usepackage[
            pdftex=true,
            bookmarks=true,        
            a4paper=true,
            pdftitle={Written with GAPDoc},
            pdfcreator={LaTeX with hyperref package / GAPDoc},
            colorlinks=true,
            backref=page,
            breaklinks=true,
            linkcolor=linkColor,
            citecolor=citeColor,
            filecolor=fileColor,
            urlcolor=urlColor,
            pdfpagemode={UseNone}, 
           ]{hyperref}

\newcommand{\maintitlesize}{\fontsize{50}{55}\selectfont}

% write page numbers to a .pnr log file for online help
\newwrite\pagenrlog
\immediate\openout\pagenrlog =\jobname.pnr
\immediate\write\pagenrlog{PAGENRS := [}
\newcommand{\logpage}[1]{\protect\write\pagenrlog{#1, \thepage,}}
%% were never documented, give conflicts with some additional packages

\newcommand{\GAP}{\textsf{GAP}}

%% nicer description environments, allows long labels
\usepackage{enumitem}
\setdescription{style=nextline}

%% depth of toc
\setcounter{tocdepth}{1}





%% command for ColorPrompt style examples
\newcommand{\gapprompt}[1]{\color{promptColor}{\bfseries #1}}
\newcommand{\gapbrkprompt}[1]{\color{brkpromptColor}{\bfseries #1}}
\newcommand{\gapinput}[1]{\color{gapinputColor}{#1}}


\begin{document}

\logpage{[ 0, 0, 0 ]}
\begin{titlepage}
\mbox{}\vfill

\begin{center}{\maintitlesize \textbf{ Jupyter-Viz \mbox{}}}\\
\vfill

\hypersetup{pdftitle= Jupyter-Viz }
\markright{\scriptsize \mbox{}\hfill  Jupyter-Viz  \hfill\mbox{}}
{\Huge \textbf{ Jupyter Notebook Visualization Tools \mbox{}}}\\
\vfill

{\Huge  1.0.0 \mbox{}}\\[1cm]
{ 26 September 2018 \mbox{}}\\[1cm]
\mbox{}\\[2cm]
{\Large \textbf{ Nathan Carter\\
    \mbox{}}}\\
\hypersetup{pdfauthor= Nathan Carter\\
    }
\end{center}\vfill

\mbox{}\\
{\mbox{}\\
\small \noindent \textbf{ Nathan Carter\\
    }  Email: \href{mailto://ncarter@bentley.edu} {\texttt{ncarter@bentley.edu}}\\
  Homepage: \href{http://nathancarter.github.io} {\texttt{http://nathancarter.github.io}}\\
  Address: \begin{minipage}[t]{8cm}\noindent
 175 Forest St.\\
 Waltham, MA 02452\\
 USA\\
 \end{minipage}
}\\
\end{titlepage}

\newpage\setcounter{page}{2}
\newpage

\def\contentsname{Contents\logpage{[ 0, 0, 1 ]}}

\tableofcontents
\newpage

     
\chapter{\textcolor{Chapter }{Function reference}}\label{Chapter_Function_reference}
\logpage{[ 1, 0, 0 ]}
\hyperdef{L}{X7ECCCA82839EA283}{}
{
  
\section{\textcolor{Chapter }{Public API}}\label{Chapter_Function_reference_Section_Public_API}
\logpage{[ 1, 1, 0 ]}
\hyperdef{L}{X784AA528839119B9}{}
{
  

\subsection{\textcolor{Chapter }{RunJavaScript}}
\logpage{[ 1, 1, 1 ]}\nobreak
\hyperdef{L}{X78BDA43D7E633E33}{}
{\noindent\textcolor{FuncColor}{$\triangleright$\enspace\texttt{RunJavaScript({\mdseries\slshape script})\index{RunJavaScript@\texttt{RunJavaScript}}
\label{RunJavaScript}
}\hfill{\scriptsize (function)}}\\
\textbf{\indent Returns:\ }
an object that, if rendered in a Jupyter notebook, will run \mbox{\texttt{\mdseries\slshape script}} as JavaScript 



 If evaluated in a Jupyter notebook, its result, when rendered by that
notebook, will run the JavaScript code in \mbox{\texttt{\mdseries\slshape script}}. 

 When the given code is run, the varible \texttt{element} will be defined in its environment, and will contain the output element in the
Jupyter notebook corresponding to the code that was just evaluated. The script
is free to write to that output element. 

 This function is not intended for use in the GAP REPL. }

 

\subsection{\textcolor{Chapter }{LoadJavaScriptFile}}
\logpage{[ 1, 1, 2 ]}\nobreak
\hyperdef{L}{X81E2B3A07EA7A02B}{}
{\noindent\textcolor{FuncColor}{$\triangleright$\enspace\texttt{LoadJavaScriptFile({\mdseries\slshape filename})\index{LoadJavaScriptFile@\texttt{LoadJavaScriptFile}}
\label{LoadJavaScriptFile}
}\hfill{\scriptsize (function)}}\\
\textbf{\indent Returns:\ }
the string contents of the file whose name is given 



 Interprets the given \mbox{\texttt{\mdseries\slshape filename}} relative to the \texttt{lib/js/} path in the Jupyter-Viz package's installation folder, because that is where
this package stores its JavaScript libraries. A \texttt{.js} extension will be added to \mbox{\texttt{\mdseries\slshape filename}} iff needed. A \texttt{.min.js} extension will be added iff such a file exists, to prioritize minified
versions of files. 

 If the file has been loaded before in this GAP session, it will not be
reloaded, but will be returned from a cache in memory, for efficiency. 

 If no such file exists, returns \texttt{fail} and caches nothing. }

 

\subsection{\textcolor{Chapter }{CreateVisualization}}
\logpage{[ 1, 1, 3 ]}\nobreak
\hyperdef{L}{X7BFFD8B7808F5BDA}{}
{\noindent\textcolor{FuncColor}{$\triangleright$\enspace\texttt{CreateVisualization({\mdseries\slshape data[, code]})\index{CreateVisualization@\texttt{CreateVisualization}}
\label{CreateVisualization}
}\hfill{\scriptsize (function)}}\\
\textbf{\indent Returns:\ }
an object that, if rendered in a Jupyter notebook, will run a script to create
the desired visualization 



 The \mbox{\texttt{\mdseries\slshape data}} must be a record that will be converted to JSON using GAP's \textsf{json} package. 

 The second argument is optional, a string containing JavaScript \mbox{\texttt{\mdseries\slshape code}} to run once the visualization has been created. When that code is run, the
variables \texttt{element} and \texttt{visualization} will be in its environment, the former holding the output element in the
notebook containing the visualization, and the latter holding the
visualization element itself. 

 The \mbox{\texttt{\mdseries\slshape data}} should have the following attributes. 
\begin{itemize}
\item  \texttt{tool} (required) - the name of the visualization tool to use. Currently supported
tools: 
\begin{itemize}
\item  \texttt{anychart}, whose JSON data format is given here:

 \href{https://docs.anychart.com/Working_with_Data/Data_From_JSON} {\texttt{https://docs.anychart.com/Working{\textunderscore}with{\textunderscore}Data/Data{\textunderscore}From{\textunderscore}JSON}} 
\item  \texttt{canvas}, that is, a regular HTML canvas element, on which you can draw using
arbitrary JavaScript included in the \mbox{\texttt{\mdseries\slshape code}} parameter 
\item  \texttt{canvasjs}, whose JSON data format is given here:

 \href{https://canvasjs.com/docs/charts/chart-types/} {\texttt{https://canvasjs.com/docs/charts/chart-types/}} 
\item  \texttt{chartjs}, whose JSON data format is given here:

 \href{http://www.chartjs.org/docs/latest/getting-started/usage.html} {\texttt{http://www.chartjs.org/docs/latest/getting-started/usage.html}} 
\item  \texttt{cytoscape}, whose JSON data format is given here:

 \href{http://js.cytoscape.org/#notation/elements-json} {\texttt{http://js.cytoscape.org/\#notation/elements-json}} 
\item  \texttt{d3}, which is loaded into an SVG element in the notebook's output cell, and the
caller can call any D3 methods on that element thereafter, using arbitrary
JavaScript included in the \mbox{\texttt{\mdseries\slshape code}} parameter 
\item  \texttt{html}, which fills the output element with arbitrary HTML, which the caller should
provide as a string in the \texttt{html} field of \mbox{\texttt{\mdseries\slshape data}}, as documented below 
\item  \texttt{plotly}, whose JSON data format is given here:

 \href{https://plot.ly/javascript/plotlyjs-function-reference/#plotlynewplot} {\texttt{https://plot.ly/javascript/plotlyjs-function-reference/\#plotlynewplot}} 
\end{itemize}
 
\item  \texttt{data} (required) - subobject containing all options specific to the content of the
visualization, often passed intact to the external JavaScript visualization
library. You should prepare this data in the format required by the library
specified in the \texttt{tool} field, following the documentation for that library cited above. 
\item  \texttt{width} (optional) - width to set on the output element being created 
\item  \texttt{height} (optional) - similar, but height 
\end{itemize}
 
\begin{Verbatim}[commandchars=!@|,fontsize=\small,frame=single,label=Example]
  CreateVisualization( rec(
    tool := "html",
    data := rec( html := "I am <i>SO</i> excited about this." )
  ), "console.log( 'Visualization created.' );" );
\end{Verbatim}
 }

 }

 
\section{\textcolor{Chapter }{Internal methods}}\label{Chapter_Function_reference_Section_Internal_methods}
\logpage{[ 1, 2, 0 ]}
\hyperdef{L}{X7B106BA97FD3C2BF}{}
{
  Using the convention common to GAP packages, we prefix all methods not
intended for public use with a sequence of characters that indicate our
particular package. In this case, we use the \texttt{JUPVIZ} prefix. This is a sort of "poor man's namespacing." 

 \emph{None of these methods should need to be called by a client of this package. We
provide this documentation here for completeness, not out of necessity.} 

\subsection{\textcolor{Chapter }{JUPVIZAbsoluteJavaScriptFilename}}
\logpage{[ 1, 2, 1 ]}\nobreak
\hyperdef{L}{X7FA27984822A0E5E}{}
{\noindent\textcolor{FuncColor}{$\triangleright$\enspace\texttt{JUPVIZAbsoluteJavaScriptFilename({\mdseries\slshape filename})\index{JUPVIZAbsoluteJavaScriptFilename@\texttt{JUPVIZAbsoluteJavaScriptFilename}}
\label{JUPVIZAbsoluteJavaScriptFilename}
}\hfill{\scriptsize (function)}}\\
\textbf{\indent Returns:\ }
a JavaScript filename to an absolute path in the package dir 



 Given a relative \mbox{\texttt{\mdseries\slshape filename}}, convert it into an absolute filename by prepending the path to the \texttt{lib/js/} folder within the Jupyter-Viz package's installation folder. This is used by
functions that need to find JavaScript files stored there. 

 A \texttt{.js} extension is appended if none is included in the given \mbox{\texttt{\mdseries\slshape filename}}. }

 

\subsection{\textcolor{Chapter }{JUPVIZLoadedJavaScriptCache}}
\logpage{[ 1, 2, 2 ]}\nobreak
\hyperdef{L}{X7F877B4279081217}{}
{\noindent\textcolor{FuncColor}{$\triangleright$\enspace\texttt{JUPVIZLoadedJavaScriptCache\index{JUPVIZLoadedJavaScriptCache@\texttt{JUPVIZLoadedJavaScriptCache}}
\label{JUPVIZLoadedJavaScriptCache}
}\hfill{\scriptsize (global variable)}}\\


 A cache of the contents of any JavaScript files that have been loaded from
this package's folder. The existence of this cache means needing to go to the
filesystem for these files only once per GAP session. This cache is used by \texttt{LoadJavaScriptFile} (\ref{LoadJavaScriptFile}). }

 

\subsection{\textcolor{Chapter }{JUPVIZFillInJavaScriptTemplate}}
\logpage{[ 1, 2, 3 ]}\nobreak
\hyperdef{L}{X84F0A3C97FBC57EE}{}
{\noindent\textcolor{FuncColor}{$\triangleright$\enspace\texttt{JUPVIZFillInJavaScriptTemplate({\mdseries\slshape filename, dictionary})\index{JUPVIZFillInJavaScriptTemplate@\texttt{JUPVIZFillInJavaScriptTemplate}}
\label{JUPVIZFillInJavaScriptTemplate}
}\hfill{\scriptsize (function)}}\\
\textbf{\indent Returns:\ }
a string containing the contents of the given template file, filled in using
the given dictionary 



 A template file is one containing identifiers that begin with a dollar sign (\texttt{\$}). For example, \texttt{\$one} and \texttt{\$two} are both identifiers. One "fills in" the template by replacing such
identifiers with whatever text the caller associates with them. 

 This function loads the file specified by \mbox{\texttt{\mdseries\slshape filename}} by passing that argument directly to \texttt{LoadJavaScriptFile} (\ref{LoadJavaScriptFile}). If no such file exists, returns \texttt{fail}. Otherwise, it proceed as follows. 

 For each key-value pair in the given \mbox{\texttt{\mdseries\slshape dictionary}}, prefix a \texttt{\$} onto the key, suffix a newline character onto the value, and then replace all
occurrences of the new key with the new value. The resulting string is the
result. 

 The newline character is included so that if any of the values in the \mbox{\texttt{\mdseries\slshape dictionary}} contains single-line JavaScript comment characters (\texttt{//}) then they will not inadvertently affect later code in the template. }

 

\subsection{\textcolor{Chapter }{JUPVIZRunJavaScriptFromTemplate}}
\logpage{[ 1, 2, 4 ]}\nobreak
\hyperdef{L}{X8603271979CCFA2A}{}
{\noindent\textcolor{FuncColor}{$\triangleright$\enspace\texttt{JUPVIZRunJavaScriptFromTemplate({\mdseries\slshape filename, dictionary})\index{JUPVIZRunJavaScriptFromTemplate@\texttt{JUPVIZRunJavaScriptFromTemplate}}
\label{JUPVIZRunJavaScriptFromTemplate}
}\hfill{\scriptsize (function)}}\\
\textbf{\indent Returns:\ }
the composition of \texttt{RunJavaScript} (\ref{RunJavaScript}) with \texttt{JUPVIZFillInJavaScriptTemplate} (\ref{JUPVIZFillInJavaScriptTemplate}) 



 This function is quite simple, and is just a convenience function }

 

\subsection{\textcolor{Chapter }{JUPVIZRunJavaScriptUsingRunGAP}}
\logpage{[ 1, 2, 5 ]}\nobreak
\hyperdef{L}{X804379D07F29E905}{}
{\noindent\textcolor{FuncColor}{$\triangleright$\enspace\texttt{JUPVIZRunJavaScriptUsingRunGAP({\mdseries\slshape jsCode})\index{JUPVIZRunJavaScriptUsingRunGAP@\texttt{JUPVIZRunJavaScriptUsingRunGAP}}
\label{JUPVIZRunJavaScriptUsingRunGAP}
}\hfill{\scriptsize (function)}}\\
\textbf{\indent Returns:\ }
an object that, if rendered in a Jupyter notebook, will run \mbox{\texttt{\mdseries\slshape jsCode}} as JavaScript after \texttt{runGap} has been defined 



 There is a JavaScript function called \texttt{runGAP}, defined in the \texttt{using-runGAP.js} file distributed with this package. That function makes it easy to make
callbacks from JavaScript in a Jupyter notebook to the GAP kernel underneath
that notebook. This GAP function runs the given \mbox{\texttt{\mdseries\slshape jsCode}} in the notebook, but only after ensuring that \texttt{runGAP} is defined globally in that notebook, so that \mbox{\texttt{\mdseries\slshape jsCode}} can call \texttt{runGAP} as needed. 

 Here is an example use, from JavaScript, of the \texttt{runGAP} function. 
\begin{Verbatim}[commandchars=!@|,fontsize=\small,frame=single,label=Example]
  var calculation = "2^50";
  runGAP( calculation + ";", function ( result, error ) {
    if ( result )
      alert( calculation + "=" + result );
    else
      alert( "There was an error: " + error );
  } );
\end{Verbatim}
 }

 

\subsection{\textcolor{Chapter }{JUPVIZRunJavaScriptUsingLibraries}}
\logpage{[ 1, 2, 6 ]}\nobreak
\hyperdef{L}{X79B7692C7F543161}{}
{\noindent\textcolor{FuncColor}{$\triangleright$\enspace\texttt{JUPVIZRunJavaScriptUsingLibraries({\mdseries\slshape libraries, jsCode})\index{JUPVIZRunJavaScriptUsingLibraries@\texttt{JUPVIZRunJavaScriptUsingLibraries}}
\label{JUPVIZRunJavaScriptUsingLibraries}
}\hfill{\scriptsize (function)}}\\
\textbf{\indent Returns:\ }
an object that, if rendered in a Jupyter notebook, will run \mbox{\texttt{\mdseries\slshape jsCode}} as JavaScript after all \mbox{\texttt{\mdseries\slshape libraries}} have been loaded 



 There are a set of JavaScript libraries stored in the \texttt{lib/js/} subfolder of this package's installation folder. The Jupyter notebook does
not, by default, know about any of those libraries. This GAP function runs the
given \mbox{\texttt{\mdseries\slshape jsCode}} in the notebook, but only after ensuring that all JavaScript files on the list \mbox{\texttt{\mdseries\slshape libraries}} have been loaded, so that \mbox{\texttt{\mdseries\slshape jsCode}} can make use of the functions and variables that they define. 

 If the first parameter is given as a string instead of a list of strings, it
is treated as a list of just one string. 
\begin{Verbatim}[commandchars=!@|,fontsize=\small,frame=single,label=Example]
  JUPVIZRunJavaScriptUsingLibraries( [ "mylib.js" ],
    "alert( 'My Lib defines foo to be: ' + window.foo );" );
  # Equivalently:
  JUPVIZRunJavaScriptUsingLibraries( "mylib.js",
    "alert( 'My Lib defines foo to be: ' + window.foo );" );
\end{Verbatim}
 }

 }

 
\section{\textcolor{Chapter }{Representation wrapper}}\label{Chapter_Function_reference_Section_Representation_wrapper}
\logpage{[ 1, 3, 0 ]}
\hyperdef{L}{X7CB82ABE7B3255EB}{}
{
  This code is documented for completeness's sake only. It is not needed for
clients of this package. Package maintainers may be interested in it in the
future. 

 The \textsf{JupyterKernel} package defines a method \texttt{JupyterRender} that determines how GAP data will be shown to the user in the Jupyter notebook
interface. When there is no method implemented for a specific data type, the
fallback method uses the built-in GAP method \texttt{ViewString}. 

 This presents a problem, because we are often transmitting string data (the
contents of JavaScript files) from the GAP kernel to the notebook, and \texttt{ViewString} is not careful about how it escapes characters such as quotation marks, which
can seriously mangle code. Thus we must define our own type and \texttt{JupyterRender} method for that type, to prevent the use of \texttt{ViewString}. 

 The declarations documented below do just that. In the event that \texttt{ViewString} were upgraded to more useful behavior, this workaround could probably be
removed. Note that it is used explicitly in the \texttt{using-library.js} file in this package. 

\subsection{\textcolor{Chapter }{JUPVIZIsFileContents (for IsObject)}}
\logpage{[ 1, 3, 1 ]}\nobreak
\hyperdef{L}{X82269732822D224D}{}
{\noindent\textcolor{FuncColor}{$\triangleright$\enspace\texttt{JUPVIZIsFileContents({\mdseries\slshape arg})\index{JUPVIZIsFileContents@\texttt{JUPVIZIsFileContents}!for IsObject}
\label{JUPVIZIsFileContents:for IsObject}
}\hfill{\scriptsize (filter)}}\\
\textbf{\indent Returns:\ }
\texttt{true} or \texttt{false} 



 The type we create is called \texttt{FileContents}, because that is our purpose for it (to preserve, unaltered, the contents of
a text file). }

 

\subsection{\textcolor{Chapter }{JUPVIZIsFileContentsRep (for IsComponentObjectRep and JUPVIZIsFileContents)}}
\logpage{[ 1, 3, 2 ]}\nobreak
\hyperdef{L}{X84D70ACD7A72AA84}{}
{\noindent\textcolor{FuncColor}{$\triangleright$\enspace\texttt{JUPVIZIsFileContentsRep({\mdseries\slshape arg})\index{JUPVIZIsFileContentsRep@\texttt{JUPVIZIsFileContentsRep}!for IsComponentObjectRep and JUPVIZIsFileContents}
\label{JUPVIZIsFileContentsRep:for IsComponentObjectRep and JUPVIZIsFileContents}
}\hfill{\scriptsize (filter)}}\\
\textbf{\indent Returns:\ }
\texttt{true} or \texttt{false} 



 The representation for the \texttt{FileContents} type }

 

\subsection{\textcolor{Chapter }{JUPVIZFileContents (for IsString)}}
\logpage{[ 1, 3, 3 ]}\nobreak
\hyperdef{L}{X7D401DB582A5EBA2}{}
{\noindent\textcolor{FuncColor}{$\triangleright$\enspace\texttt{JUPVIZFileContents({\mdseries\slshape arg})\index{JUPVIZFileContents@\texttt{JUPVIZFileContents}!for IsString}
\label{JUPVIZFileContents:for IsString}
}\hfill{\scriptsize (operation)}}\\


 A constructor for \texttt{FileContents} objects }

 Elsewhere, the \textsf{Jupyter-Viz} package also installs a \texttt{JupyterRender} method for \texttt{FileContents} objects that just returns their text content untouched. }

 }

 \def\indexname{Index\logpage{[ "Ind", 0, 0 ]}
\hyperdef{L}{X83A0356F839C696F}{}
}

\cleardoublepage
\phantomsection
\addcontentsline{toc}{chapter}{Index}


\printindex

\immediate\write\pagenrlog{["Ind", 0, 0], \arabic{page},}
\newpage
\immediate\write\pagenrlog{["End"], \arabic{page}];}
\immediate\closeout\pagenrlog
\end{document}
